\documentclass[11pt]{article}
\usepackage[utf8]{inputenc}
\usepackage[spanish]{babel}
\usepackage{fancyhdr}
\usepackage{amsmath}
\usepackage{algorithm,algorithmic}
\usepackage{amsfonts}
\usepackage{amssymb}
\usepackage{makeidx}
\usepackage{graphicx}
\usepackage{fourier}
\usepackage{geometry}
\pagestyle{fancy}
\fancyhf{}
\rhead{Documento1}
\lhead{Algoritmos Distribuidos}
\rfoot{Page \thepage}
\author{Erick}
\newtheorem{theorem}{Theorem}
\title{Elementos de un Algoritmo Distribuido.}
\begin{document}
\begin{titlepage}
    \begin{Large}
        Analisis de un Algoritmo en un Ambiente Distribuido.
    \end{Large}
\begin{abstract}
Estudiamos y presentamos los elementos del Modelo de Computo
Distribuido, en el cual se  definió de manera formal sus elementos, asi como se formalizo lo que significa la ejecución del algoritmo en un modelo de computo Distribuido
\\
Se presentaran y definirán las nociones del análisis de un Algoritmo Distribuido, como son:
\begin{itemize}

 \item Complejidad del Algoritmo,i.e Complejidad temporal.
 \item La exactitud o correccion del Algoritmo.
\end{itemize}

Entonces con estas nociones presentaremos un problema en el contexto de un ambiente distribuido
el cual sera resulto diseñando un algoritmo.
\end{abstract}
\end{titlepage}
\begin{center}
    \begin{Large}
    \textit{Definiciones y Conceptos}
    \end{Large}
\end{center}

\textbf{\emph{Definicion de complejidad de un Algoritmo ejecutado en un modelo de computo: $ LOCAL $.}}\\


Como en un modelo secuencial la medida de complejidad en el sentido temporal es el numero de pasos que toma del inicio al final, entonces inspirado en esa noción de complejidad secuencial definimos el siguiente concepto \\

\textbf{\emph{Complejidad Temporal de un Algoritmo}}\\

\emph{Definición:}

Sea $ G $ una gráfica y $ \pi $ un algoritmo entonces definimos la complejidad de $ \pi $ de manera \emph{Sincrona} como
el numero de rondas que se generaron durante la ejecución de $ \pi $ en $ G $ \\

De igual manera se puede definir la complejidad de una manera \emph{Asincrona} como las unidades de tiempo que se generan desde el inicio del algoritmo $ \pi $ en $ G $ hasta el final de su ejecución.\\


\textbf{\emph{Noción de exactitud o correcion del Algoritmo.}}

\emph{Definición}\\

Diremos que un Algoritmo es correcto si:
\begin{itemize}
       \item Resuelve el problema computacional por el cual fue diseñado
       \item Para cada entrada, produce la salida deseada
       \item Termina en un tiempo de ejecución finito

\end{itemize}
\emph{Definicion(Notacion Big-Oh):} \
Si $f$ y $g$ son funciones de $\mathbb{Z}$ en $\mathbb{Z}$ entonces decimos que:\
\begin{itemize}
    \item $f = O(g)$ si existe una constante $c$ tal que:
    $f(n) < c\dot g(n)$ para todo $n$ suficientemente grande.
    \item Diremos que $f = \theta(g)$ si $f = O(g) y g = O(g)$
\end{itemize}

Entonces con base a esa definición podemos encapsular la complejidad(Temporal)
del Algoritmo en términos de las definiciones anteriores,
entonces podemos decir que la complejidad esta dada de la siguiente manera:
 \emph{ como la complejidad temporal es dada con sintaxis y semántica}:\\

$ Time( \pi, G) = O(g(n) $ donde $ g $ dependerá de la naturaleza del algoritmo en cuestión.\\
\newpage
\begin{center}
    \begin{Large}
    \textit{Presentacion del Problema a resolver y su respectivo Algoritmo}
    \end{Large}
\end{center}
\textbf{ Tenemos en contexto el siguiente Problema:}
\\
Dado una red de procesadores con cierta topologia, a saber la topologia de un arbol,
donde cada nodo tiene un entero $n$, entonces el problema a solucionar es encontrar el
elemento mas repetido en toda la red, i.e encontrar estadisticamente el modo de la red;
para eso debemos implementar un protocolo de comunicacion en el Arbol para determinar dicho valor.
Que en terminos formales es diseñar un algoritmo que sera una tupla de $n$ protocolos los cuales
se ejecutaran en el Arbol para solucionar dicho problema.

\textbf{Definicion}
Conceptualmente tendremos los siguientes tipos de mensajes que se enviaran en el protocolo que se enuciara
mas adelante:

\
\begin{itemize}
    \item Mensajes del tipo 1: $M(\uparrow, A_{i1},A_{i2}) $
    \item Mensajes del tipo 2: $ M(\uparrow,(x_{modo}), max(A_{i2}))$
\end{itemize}
\
Donde $A_{i1}$ es un arreglo de los valores locales de los nodos del sub-arbol enraizado
en el nodo $p_{i}$ y $A_{i2}$ es un arreglo de las frecuencias de los valores locales en $A_{i1}$
\\

\textbf{Definicion}\\
Sean $M\uparrow, A_{i1}, A_{i2})$, $M(\uparrow, A_{k1}, A_{k2})$ mensajes del tipo 1, entonces podemos definir la operacion $Sum(M(\uparrow, A_{i1}, A_{i2}), M(\uparrow, A_{k1}, A_{k2}))$ y la definimos
de la siguiente manera:\\
\begin{enumerate}
    \item Sea $x$ en $A_{i1}$, si $x$ es tal que es distinto a todo $z$ en $A_{j1}$
     entonces \\ $A_{r1} = A_{i1}[y] \star A_{k1}$ y $A_{r2} = A_{i2}[y] \star A_{k2}$ \\
     donde $\star$ denota la concatenacion del array formado por el elemento  $x = A_{i1}[y]$ con $A_{k1}$ y tambien denota
     la concatenacion del elemento $f_{x} = A_{i2}[y]$ con el array $A_{k2}$, es decir estamos generando un array
     que contiene a los que estamos operando bajo $\star$
    \item Sea $x$ en $A_{i1}$, si es tal que $x = A_{i1}[y] = A_{k1}[y']$ entonces $A_{r1}[y''] = A_{i1}[y] = A_{i2}[y']$  y
    $A_{r2}[y''] = A_{i2}[y] + A_{k2}[y']$
\end{enumerate}

Y denonamos a $Sum(M(\uparrow,A_{i1}, A_{i2}), M(\uparrow, A_{k1}, A_{k2})) = M(\uparrow, A_{r1}, A_{r2})$

\newpage

En las siguientes lineas se muestra el Algoritmo.


\begin{algorithm}
\begin{algorithmic}
    \STATE $A_{first}( p_{i}, ID, x_{i} \in \mathbb{N})$
    \IF { $p_{i}$ es una hoja}
      \STATE $A_{r1} = (x_{p_{i}})$
      \STATE $A_{r2} = (1)$
      \STATE \textbf{send} $M_{p_{i}} = M(\uparrow, A_{r1}, A_{r2})$ a su padre
    \ELSE
      \STATE \textbf{wait} $M_{1},\dots,M_{n}$ mensajes de sus hijos
      \STATE  $M_{j} = M(\uparrow, A_{j1}, A_{j2})$
      \FORALL {$l,s \in \lbrace 1,\dots,n+1 \rbrace$ con $l \neq s$}
        \STATE $Sum(M(\uparrow, A_{l1}, A_{l2}), M(\uparrow, A_{s1}, A_{s2}))$
      \ENDFOR
      \STATE $M(\uparrow, A_{r1}, A_{r2}) = Sum(M(\uparrow, A_{11}, A_{12}), \dots, M(\uparrow, A_{n1}, A_{n2})) $
      \STATE $A_{p_{i}1} = A_{r1}$ // Es el array después de las operaciones de los mensajes de sus Hijos
      \STATE $A_{p_{i}2} = A_{r2}$ // Es el array después de las operaciones de los mensajes de sus Hijos
      \IF { $p_{i}$ es la raiz}
        \STATE $M_{p_{i}} = M(\downarrow, (x_{modo}), max(A_{r2}))$
        \STATE \textbf{return} $x_{modo}$
      \ELSE
        \STATE \textbf{send} $M_{p_{i}} = M(\uparrow, A_{r1}, A_{r2})$ a su padre
      \ENDIF
    \ENDIF
\end {algorithmic}
\caption{First-dis($T_{r_0},ID,x_{i}$)\label{lss}}
\end{algorithm}

    \newpage
    \begin{center}
        \begin{Large}
            \textit{Correccion del Algoritmo y complejidad de los algoritmos.}
        \end{Large}
    \end{center}

En esta parte del documento se mostrara que en efecto $A_{first}$ hace lo que tiene que hacer, que en terminos
formales es decir que $A_{first}$ es correcto, para ello diremos que es invariante en el siguiente sentido:
Que para cada ronda $ r \in \lbrace 1,\dots, h_{r_{0}} \rbrace $ se cumple que el nodo $v$ de altura $h_{v}$ ha calculado correctamente
su variable local a saber su correspondiente $M_{v}$ que es un mensaje de la lista de tipos.
\\

\begin{theorem}
Sea $ \pi = A_{first}$ entonces tiene la propiedad de que para cada $r \in \lbrace 1, \ldots, h_{r_{0}}  \rbrace $
al final de la ronda r el nodo $v$ de altura $h$ ha calculado correctamente su variable local $M_{v}$ y
ademas en el sub-arbol $T_{v}$ que denota que esta enraizado en $v$ tiene en su variable local las entradas de los nodos de $T_{v}$
y con posibles repeticiones.

    \textbf{Demostracion:} \\
Sea $h=1$ entonces al final de la ronda correspondiente, $v$ es una hoja con
base al init del algoritmo\\
$\therefore M_{v_{h=0}}= M(\uparrow, A_{r1}, A_{r2})$

Suponemos que es cierto para el paso $h=j$  y demostremos que es cierto para $h = j+1$
como es cierto para el paso $h =j$ entonces tiene el subarbol $T_{v_{j}}$ calculada la variable
local correspondiente, luego por la construccion de $A_{first}$ se tiene que en $v$ de altura $j+1$
tambien tiene calculada su variable local que es por la construcción del algoritmo es:
\\
$\therefore {v_{h = j+1}} = M(\uparrow, A_{r1}, A_{r2})$
\end{theorem}
Del razonamiento anterior se desprende que el nodo que calcula el maximo es en efecto la raiz del Arbol

\begin{theorem}
    Para el Algoritmo $A_{first}$ en la ronda $r$ el nodo $v$ esta calculando de manera implicita el elemento mas repetido,
    mas aun en la ronda $r = h_{r_{0}}$ es el que toma la decision final de cual es elemento mas repetido y disemina
    el mensaje a manera de reponse en el resto del arbol

    \textbf{Demostracion:}\\
    Sea $r$ una ronda, entonces notamos que en la correspondiente ronda el vertice $r$ podemos hacer la operacion
    en la variable local $M_{v}$ que tiene una representacion $M(\uparrow, A_{r1}, A_{r2})$ por las lineas de codigo del
    algoritmo, en particular  podemos calcular $max(A_{r2})$ que por el mapeo implicito entre $A_{r1}$ y $A_{r2}$ le corresponde
    el elemento mas frecuente en $A_{r2}$ a saber: $x_{modo}$
    Y observamos que si $h= h_{r_{0}}$ y seguimos las lineas de codigo para ese caso, entonces para $x_{modo_{r_0}}$ es el valor
    que se toma finalmente y es en la ronda en la que se  disemina dicho mensaje.

\end{theorem}


    Ahora tenemos que analizar y deducir la complejidad del Algoritmo en el sentido temporal y de mensajes.
    Por un lado la complejidad temporal se puede desprender observando la demostracion del teorema anterior:
    que para la altura correspondiente a la raiz denotada como $h_ {r_{0}}$ sera la complejidad temporal pero solo en
    la parte del algoritmo \textbf{request}, pues en el \textbf{response} tardara exactamente lo mismo hasta diseminar
    el mensaje en el resto de los nodos del arbol, entonces lo decimos mas formalmente en el siguiente enunciado:

\begin{theorem}
    Sea $\pi = A_{first}$ entonces la complejidad temporal denotada como $Time(T_{first})$ = $O(h_{r_{0}})$
    y la complejidad de mensajes es exactamente $Message(A_{first}) = O(m)$ donde $m$ denota el numero de aristas
    en $T_{r_{0}}$


   \textbf{Demostracion:}
\begin{flushleft}
    Por construccion del algoritmo y por la correccion del algoritmo, en particular por la propiedad de Loop-Invariant
    podemos observar que se disemina el mensaje hasta el paso de la induccion que es $h_{r_{0}}$ correspondiente a la altura de
    de la raiz, y luego en el proceso de diseminar el mensaje en todo el arbol de manera de \textbf{response}
    es la misma altura, entonces la complejidad temporal es $O(h_{r_{0}})$

    Por otro lado, afirmamos que hay tantas rondas como canales de comunicacion, pues en esencia estamos mapeando
    la cantidad de mensajes con los canales de comunicacion, pero eso solo en el proceso de \textbf{request} por el diseño del
    algoritmo, en el proceso de \textbf{response} que es la diseminacion del mensaje es el mismo mapeo que en el proceso
    de \textbf{request} entonces hay tantas rondas como el doble de canales de comunicacion en el proceso de \textbf{request, response},
    mas formalmente lo escribimos de la siguiente manera:
    $Message(A_{first}) = O(m)$ donde $m$ denota el numero de aristas en en el Arbol en el que estamos ejecutando el algoritmo.

\end{flushleft}
\end{theorem}

    \textbf{Observacion:}\\
    Podemos obsevar que existe un subarbol en el que su raiz toma la decision de hacer:\\
    \textbf{return} $x_{modo}$\\
    Es decir podemos decir que se puede optimizar $A_{first}$ sujeto a la altura $h$, y apartir de ese subarbol diseminar
    el mensaje a manera de \textbf{response} en todo el arbol $T_{r_{0}}$

\end{document}