%! Author = erick-hdz
%! Date = 10/01/20

% Preamble
\documentclass[10pt]{article}

% Packages use the same packages
\usepackage{amsmath}
\usepackage[utf8]{inputenc}
\usepackage[spanish]{babel}
\usepackage{fancyhdr}
\usepackage{graphicx}
\usepackage{algorithm,algorithmic}
\usepackage{amsfonts}
\usepackage{amsbsy}
\usepackage{makeidx}
\usepackage{geometry}
\usepackage{listings}
\usepackage[T1]{fontenc}
\usepackage{natbib}
\usepackage{amsthm}
\usepackage{amssymb}

% Status
\newtheorem*{corrolary}{Corollary}
\newtheorem*{remark}{Remark}
\newtheorem{theorem}{Theorem}[section]
\newtheorem{definition}{Definition}
\pagestyle{fancy}
\fancyhf{}
\rhead{Avance de Tesis}
\lhead{Comparación de Modelos}
\rfoot{Page \thepage}
\author{Erick Hernandez Navarrete}
\title{Simulación de Modelos}

% Document
\begin{document}
    %% En esta parte es solo la continuacion de la parte anterior
    \begin{titlepage}
        \begin{Large}
            Segunda Parte del avance de Tesis.
        \end{Large}
    \end{titlepage}

    \begin{abstract}
        Expondremos las nociones y atributos de los conceptos que nos permita
        hacer la simulación entre el mundo distribuido y el mundo de las maquinas
        de Turing.
    \end{abstract}
    \space
    \begin{center}
        \textbf{Definiciones}
    \end{center}
    En esta parte del documento vamos a formalizar las nociones, de las cuales se desprenderá casi de manera
    inmediata la subrutina que dara el control en la estructura recursiva.\\
    Entonces definamos las nociones de la siguiente manera:
    \\
    Una vez que tenemos la maquina de Turing como modelo formal en Computación, entonces con los elementos que tiene la
    maquina de Turing y el modelo particular $LOCAL$ de computo Distribuido haremos lo siguiente, definamos la noción de localidad de la siguiente manera:

    \theoremstyle{definition}
    \begin{definition}
        Sea $v \in VG$ un proceso en el modelo Formal \textbf{LOCAL}, vamos a denotar lo siguiente:
        $(w_{i},p)$ que va a denotar que el elemento de la cadena $w_{i}$ tiene localidad $v$,
        o en otros términos, que a $v$ le pertenece el elemento de localidad $i$.
    \end{definition}
    De aquí se desprende la relación que existe entre los elementos de la cadena y los procesos
    del modelo $\textbf{LOCAL}$

    \theoremstyle{definition}
    \begin{definition}
        Sean $v \in VG$ y $w_{i} \in w$ un proceso y un elemento de la cadena que es aceptada por $TM$
        entonces decimos que están relacionados si y solo si $(w_{i},v)$, i.e si y solo si $w_{i}$ tiene localidad $v$
    \end{definition}
    \\
    \begin{remark}
        Una vez que tenemos esas nociones podemos construir un conjunto para cada proceso $v$ en $LOCAL$:\\
        $LOC(P) = \{w_{i} \in w : (w_{i},P)\}$  que es generada por la relación definida anteriormente,
        este conjunto esta encapsulando las nociones anteriormente descritas.
    \end{remark}

    %Recordar la arquitectura de memoria, y mejorar la formalizacion de la idea
    %De localidad, que no es mas que la etiqueta de pertenencia
    Entonces una vez que tenemos esto bien definido, podemos hacer lo siguiente:\\
    De manera local sea $(w_{i},v)$ entonces en un sentido computacional y de entrada como un sistema computacional
    se traduce a $v(w_{i})$.
    \begin{titlepage}
    \textbf{Descripción del Diseño del Algoritmo}
    \end{titlepage}
    \space
    Una vez que tenemos la pareja entonces tenemos de manera computacional lo siguiente $v(w_{i})$ que es el
    procedimiento computacional que hemos de desarrollar tal que solucione nuestro problema con la entrada
    de localidad asignada, entonces lo que haremos es definir primero una instancia de $\delta$ de tal manera
    que cuando hagamos la llamada a delta esta llamada sera iterativa hasta que la localidad de la cadena que nos
    de la subrutina delta nos de una localidad distinta al que tiene el proceso actual entonces tendremos un mensaje
    $M$ o en su defecto acabara su computo en esa ronda, pero si no lo que hará es enviar es el estado que produjo
    y la cadena que nos genero en esa ronda $r$ el proceso $p$.\\
    Pero para ello construiremos una subrutina Booleana que lo que hará es un token en las entradas via nuestra noción
    de localidad.
    Una vez que tenemos las estructuras bien definidas hacemos lo siguiente:
    \\
    \begin{algorithm}
        \begin{algorithmic}
            \IF{$(w_{i},p)$}
              \RETURN{true}
            \ELSE
              \RETURN{false}
            \ENDIF
        \end{algorithmic}
        \caption{$Func\char95 Boolean(w_{i},p)$}
    \end{algorithm}
    \\
    Básicamente lo que esta haciendo es dar una estructura booleana que sera la afirmación
    que dara el control en el siguiente algoritmo.
    Una vez que tenemos la subrutina de tipo Booleana, diseñemos el algoritmo principal del cual demostrara que la cadena $w$ es una
    cadena aceptada por el modelo $\textbf{LOCAL}$, y esto a un alto nivel demostrara nuestro teorema principal.
    \\\\
    \begin{algorithm}
        \begin{algorithmic}
            \STATE $w \gets (w_{1}, \mathellipsis, w_{n})$
            \STATE \textbf{Síncronamente}
            \FORALL{$r$ \TO{to} $m$}
                    \STATE $p(w_i,\delta())$
                    \STATE $q \gets q_{0}$
                    \WHILE{$func\char95 boolean(w_{i},p)$}
                      \STATE \textbf{call} $\delta(q,w_{i}) \gets (q_{R},w_{R},Pos)$
                    \ENDWHILE
                    \IF{$q_{R}=q_{accept}$}
                       \RETURN{$q_{accept}$}
                    \ELSE
                       \STATE $M = <q_{R},w_{R}, Pos>$
                       \STATE \textbf{send} $M$
                    \ENDIF
            \ENDFOR
        \end{algorithmic}
        \caption{$Simulate\char95 Algo \char95 TM(w,G,TM)$\label{lss}}
    \end{algorithm}
    \\\\
    \begin{center}
       Ejemplo
    \end{center}
    Podemos exponer de manera concreta la siguiente situación: \\ tomar una linea de procesos, de tal manera que el pariente de $i$ es $i+1$, para $i$ en el
    conjunto de indices de los procesos de $V(G)$, y mas aun el pariente de $n$ es $1$.\\
    Entonces lo que podemos observa es que tenemos computacionalmente es que la subrutina
    $func \char95 boolean()$ es el que da el control del computo que tienen los procesos del modelo $LOCAL$,
    asi que este ejemplo nos da una noción natural de lo que esta ocurriendo por que en particular se tiene:\\
    $(w_{i},p_{j})$ si y solo si $i=j$, esta asignación es por construcción de la topología de la gráfica,
    esto nos da una noción geométrica de lo que computacionalmente esta ocurriendo en este algoritmo distribuido.
    \\
    Lo que prosigue es la demostración de la corrección del algoritmo.
    \\

    % Here we continue with the theorems that prove the correctness of the
    %!Anuncion de los teoremas Centrales
    \begin{theorem}
        El algoritmo $Simulate\char95 Algo\char95 TM$ es correcto.
    \end{theorem}
    \begin{proof}
        Sea $i=r$ una roda en la ejecución del algoritmo.\\
        En esa ronda tenemos la siguiente estructura de dato $M = <w_{R}, q_{R}, P>$ donde
        $P\in {LEFT,RIGHT,STAY}$ para esa ronda tenemos que el $w_{R}$ que es el resultado de la
        llamada recursiva de $\delta$, podemos concluir que $func-boolean(w_{r},p) = false $
        por lo tanto no es de la localidad de $p$, entonces podemos tomar las siguientes decisiones:
        como $w_{R}\in w$ es decidible en el sentido de la maquina de Turing, entonces $q_{R}$ puede que
        sea el estado de aceptación, en ese caso el algoritmo toma la decision regresar $q_{R}$ en su defecto
        toma la decision de enviar el mensaje $M$ a sus vecinos,
        para el cual como $TM$ es tal que $w$ es decidible entonces sin perdida de generalidad en la ronda $r=i+1$ se tendrá el estado de aceptación
        para un $p$ proceso $p$ con el ambiente de $LOCAL$.\\
        Por lo tanto existe un proceso $p$ tal que regresa un estado de aceptación.\\
        $\therefore w \in L(Simulate\char95 Algo\char95 TM)$.\\
        $\therefore Simulate\char95 algo\char95 TM()$ es correcto.
    \end{proof}

    \begin{corrolary}
        $\forall w \in L(TM)$ $\exists \pi$ algoritmo en el modelo $LOCAL$ tal que
        $w\in L(\pi)$.
    \end{corrolary}
    \begin{proof}
        Por el teorema anterior hemos construido un algoritmo en $LOCAL$ tal que
        hace decidible a $w$, i.e $w \in L(Simulate\char95 Algo\char95 TM)$(En el sentido distribuido).\\
        Mas aún como $w$ es tal que $w \in L(TM)$ entonces podemos concluir que $\pi = Simulate\char95 Algo\char95 TM$.\\
        $ \therefore \forall w \in L(TM), w\in L(\pi)$.
    \end{proof}
    %!Con esto hemos terminado el teorema Principal de esta tesis.

\end{document}
