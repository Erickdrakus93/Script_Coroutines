%! Author = erick-hdz
%! Date = 21/02/20
% This is the corrections of the father of the last file
% Preamble
\documentclass[11pt]{article}
% Packages (Here is the same structure of the father script)
\usepackage{amsmath}
\usepackage[utf8]{inputenc}
\usepackage[spanish]{babel}
\usepackage{fancyhdr}
\usepackage{amsfonts}
\usepackage{amsbsy}
\usepackage{makeidx}
\usepackage{listings}
\usepackage{geometry}
\usepackage[T1]{fontenc}
\usepackage{natbib}
%Status
\newtheorem*{remark}{Remark}
\theoremstyle{definition}
\newtheorem{definition}{Definition}[section]
\pagestyle{fancy}
\fancyhf{}
\rhead{Avances de tesis version 2}
\lhead{Comparacion de Models y Formalizacion}
\rfoot{page \thepage}
\author{Erick Hernandez Navarrete}
\title{Simulaciones de los modelos TM vs DC}
% Document
\begin{document}
    %Todo: Mejorar la calidad anterior para que tenga una buena
    %Todo: Presentacion de las cosas que hemos echo hasta esta parte
    %TODO: This is init part is essentially the same as his father, so we
    %TODO: Do the same as the main object as the part of the main example
    \begin{titlepage}
        \begin{large}
            Correcciones del Documento padre y las formalizaciones.
        \end{large}
    \end{titlepage}

    \begin{abstract}
        Aqui vamos a formalizar las ideas que hemos expuesto en el documento
        padre de este mismo, y vamos a dar los teoremas que demuestran la correccion del
        algoritmo que estas ideas construyen.
    \end{abstract}
    \newline
    En esta parte del Documento, vamos abordar el problema formalizando las nociones
    que van a hacer que tengamos una subrutina que es el que va a dar el control en
    la estructura recursiva, entonces definamos dichas nociones de la siguiente manera:
    \space
    Sea una maquina de Turing $MT$ y una cadena que es aceptada por dicha maquina de Turing, con los atributos que tiene la maquina de Turing,
    Entonces definimos la nocion de localidad en $DC_{LOCAL}$ de la siguiente manera:
    \begin{definition}{Localidad en $\psi$}
        Sea $v \in VG$ un proceso en el modelo Distribuido,
        vamos a denotar lo siguiente:
        \begin{equation}
            (w_{i},v)
        \end{equation}
        esto va a denotar que el elemento de la cadena de $\psi$ tiene localidad $v$
    \end{definition}

    \begin{remark}
        Observemos que de la definicion anterior se desprende una relacion entre los elementos
        de $\psi$ tal y como lo observamos en la siguiente definicion
    \end{remark}

    \begin{definition}
        Sean $v\in VG$ y $w_{i} \in \psi$ entonces decimos que estan relacionados
        si y solo si $(w_{i}, v)$
    \end{definition}
    \newline
    Basicamente lo anterior lo que esta haciendo es hacer una formalizacion de las nociones
    que tenemos en los documentos anteriores.
    \newpage
    Entonces una vez que tenemos esto bien definido, podemos hacer lo siguiente:
    De manera local sea $(w_{i},v)$ entonces hacemos lo siguiente $v(w_{i})$ que es la asignacion que tiene
    sentido entonces podemos hacer lo siguiente:
    \textbf{Descripcion del Diseño del Algoritmo}
    Una vez que tenemos la pareja entonces tenemos de manera computacional lo siguiente $v(w_{i}$ que es el
    procedimiento computacional que hemos de desarrollar tal que solucione nuestro problema con la entrada
    de localidad asignada, entonces lo que haremos es definir primero una instancia de $\delta$ de tal manera
    que cuando hagamos la llamada a delta 
\end{document}