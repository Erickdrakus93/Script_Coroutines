%%
%% Author: erick
%% 06/11/2019
%% Here we can use the init form of the main theorem

% Preamble
\documentclass[11pt]{article}


% Packages
\usepackage[utf8]{inputenc}
\usepackage[spanish]{babel}
\usepackage{fancyhdr}
\usepackage{amsmath}
\usepackage{algorithm,algorithmic}
\usepackage{amsfonts}
\usepackage{amssymb}
\usepackage{makeidx}
\usepackage{graphicx}
\usepackage{fourier}
\usepackage{geometry}
\usepackage{listings}

% Status
\pagestyle{fancy}
\fancyhf{}
\rhead{Avance de tesis}
\lhead{Comparacion de modelos}
\rfoot{Page \thepage}
\author{Erick} % Previous here we defined in the sort.
\newtheorem{theorem}{Theorem}
\title{Simulaciones de Modelos}

% Document
\begin{document}

    \begin{titlepage}
        \begin{Large}
            Resultados de la Comparacion de los Modelos de Computo
        \end{Large}
    \end{titlepage}

    \begin{abstract}
        Mostraremos los conceptos que nos permitiran enunciar los
        teoremas centrales que desprenderan los problemas computacionales en Cuestion
        asi como su solución.
    \end{abstract}

    En esta parte seremos muy breves en que es los conceptos que son elementales
    para la solucion y presentacion de los problemas Computacionales en Cuestion.
    \\
    \begin{center}
        \begin{Large}
            Definiciones y conceptos
        \end{Large}
    \end{center}
    \\
    \textbf{\emph{Caracteristicas del Modelo de Computo Distribuido LOCAL}}
    \\
    \begin{itemize}
        \item Cada proceso se inicializa en el mismo tiempo
        \item Los eventos Comunicacion y Computo se ejecutan en rondas y de manera sincronizada
    \end{itemize}
    \\
    \textbf{\emph{Nocion de solucionar un problema con un algoritmo Distribuido}}\\
    %% TODO: Definir que es un algoritmo Correcto, para invocarlo al definir que significa que
    %% TODO: ... que un problema sea resuelto por un un algoritmo en un sentido usual.
    \textbf{Definicion}
    Decimos que un algoritmo $\pi$  en el modelo \emph{LOCAL} es correcto con respecto a un problema $P$
    si para cada instancia de la entrada $I$ ocurre que  $\pi(I) = P $\\
    \textbf{Definicion}\\
    Entonces decimos que un algoritmo soluciona a $P$ si existe un algoritmo $\pi()$ correcto con
    respecto al problema en cuestion.
    \\
    % TODO: Falta la definicion de la extension de Decidible en el modelo
    % TODO: Distribuido y formalizar lo que significa que un algoritmo se puede replicar en otro.

    \textbf{\emph{Extension de la noción de Decibilidad en el sentido Distribuido}}\\
    \textbf{Definicion}\\
    Sea una cadena $x = x_{1}\mathellipsis x_{n}$, entonces decimos que dicha cadena es aceptada
    por un algoritmo $\pi$ si al inicializar cada proceso $v_{i}(x_{i})$, donde $i = 1,\mathellipsis,n$,
    al finalizar el computo al menos un proceso \emph{acepta} localmente.
    %TODO: Mejorar en la definicion, tomando en cuenta la existencia del proceso
    %TODO: Que hace que tenga convergencia en el sentido usual de maquinas de Turing
    \\

    %Aqui enunciremos al mismo tiempo que es que una maquina de Turing sea simulada por un algo
    %-ritmo en el sentido Distribuido y reciprocamente.
    Una vez que tenemos en mente las nociones anteriores es entonces cuando nos emprendemos
    a hacer una conexion entre ambos mundos en el sentido de que podemos introducir la nocion de
    simulacion en un sentido formal en teoria de la Computacion
    \\
    \textbf{\emph{Definicion}}\\
    Sean $MT$ y un algoritmo distribuido (En el sentido en el que presentamos la extension del modelo
    general) i.e $\pi$ y los lenguajes que deciden estos dos objetos, i.e $L(TM)$ y $L(\pi)$
    entonces decimos que la maquina de Turing simula el algoritmo distribuido si $L(TM) = L(\pi)$
    y reciprocamente.\\

    Una vez que tenemos definido que es la simulacion en un sentido formal, es entonces cuando podemos
    anunciar los teoremas de Conexion con todos los ingredientes de desarrollo de estos Objetos.

    %TODO: Declarar los teoremas que enuncian la conexion de los modelos de Estudio

    Por un lado queremos decir que si tomamos una maquina de Turing entonces existe un $\pi$ que
    simula a dicha maquina de Turing diganos $TM$ \\

    Y por otro lado queremos decir que si tomamos un algoritmo $\pi$
    este algoritmo puede ser simulado por la maquina de Turing $TM$\\

    Entonces declaremos lo anterior en Teoremas, de la siguinte manera:
    \\
    %TODO: Hacer que el teorema anterior se traduzca a un problema Computacional
    \begin{theorem}
        $\forall TM $ Maquina de Turing con $L(TM)$ entonces
       $ \exists \pi$ tal que $L(\pi) = L(TM)$
    \end{theorem}
    %Teorema_2 que es el reciproco al teorema anteriormente enunciado
    \begin{theorem}
        $\forall \pi$ Algoritmo Distribuido con $L(\pi)$ el Lenguaje que decide dicho
        algoritmo, entonces $\exists TM$ tal que $L(TM) = L(\pi)$
    \end{theorem}

    %TODO:Dar propuestas de Algoritmos y programas en el sentido de TM tal que
    %TODO: Den una demostracion de un lado de la demostracion.
    \\
    Iniciaremos con el primer Teorema, entonces para ello, tomamos una maquina de Turing,
    y lo que queremos es exhibir que existe un algoritmo $\pi$ distribuido tal que simula a $TM$

    % TODO: Mostrar el Pseudocodigo del Algoritmo que soluciona el problema por un lado.
    % Esto se reduce a diseñar el algoritmo Distribuido que solucine el problema en Cuestion.
    \\

    \begin{algorithm}
        \begin{algorithmic}
            \STATE $\pi(x_{1}\mathellipsis x_{n})$
            \FORALL {$r=0$ \textbf{to} $n$}
               \STATE \textbf{init} $v_{i}(x_{i})$
                 \IF {$x_{i} \in L(TM)$}
                    \STATE \textbf{return} $q_{accept}$
                 \ELSE
                    \STATE \textbf{send} $ x_{i}$
                 \ENDIF
            \ENDFOR
            \STATE \textbf{finally} \textbf{return} $q_{accept}$
        \end{algorithmic}
        \caption{$\pi(x_{i}\mathellipsis x_{n}$)\label{lss}}
    \end{algorithm}

    \\
    En este Diseño del Algoritmo $\pi(x)$ donde $x = x_{1},\mathellipsis, x_{n}$
    es una cadena que pertenece al $L(TM)$ que es el lenguaje aceptado por
    la maquina de Turing, ya que por un lado del Teorema Tenemos todos los elementos
    de la maquina de Turing.
    \\
    En el siguiente Teorema Encapsulara una de las cuestiones que tiene el algoritmo
    con respecto a como se esta haciendo el control de la cadena sobre las entradas de
    las subrutinas que tienen comunicación una a la otra para solucionar el problema.
    % TODO: Mostrar en que momentos puede existir una falla en el programa Anterior.
    \\
    \begin{theorem}
        Sea una cadena $x$, entonces dicha cadena es aceptada por la maquina 
        de Turing si y solo si es aceptada por $\pi$. Donde $\pi$ es un algoritmo
        Distribuido.
    \end{theorem}

    \newpage
    Con base al Diseño del Algoritmo anterior y al teorema Anterior podemos,
    encontrar una cadena que es decidible por $TM$ pero que no es decidible
    con nuestro algoritmo descrito anteriormente.
    \\
    El siguiente programa hace que se solucione esta cuestion:

    % Esta es la subrutina que hemos de describir de manera trivial como sigue:
    \begin{algorithm}
        \begin{algorithmic}
            \FORALL {$r \leftarrow 1$ \textbf{to} $n$}
            \STATE $\pi(x_{1},\mathellipsis, x_{n})$
            \STATE \textbf{init} $p_{i}(x)$
               \STATE \textbf{return} $q_{accept}$
            \ENDFOR
        \end{algorithmic}
        \caption{$\pi(x,G)$\label{lss}}
    \end{algorithm}
    \\

    Podemos observar que esta subrutina hace que el problema se trivialice;
    ya que estamos alimentando a cada proceso con toda la entrada que come
    $\pi$, lo cual rompe con el paradigma Distribuido, ya que todos los procesos
    tiene toda la informacion.

    %TODO: Hacer de las subrutinas anteriores una subroutina tal que
    %TODO: Podamos hacer una primitiva con la operacion Broadcast de la cual
    %TODO: Podamos hacer de los algoritmos anteriores primitivas

    Sea una gráfica de diametro D, podemos diseñar el siguiente algoritmo
    solucionando de manera correcta el problema en cuestión $P$ reduciendo
    un grado de libertad con la restricción anterior sobre el diametro de
    la gráfica, y con los dos programas anteriores como subrutinas podemos
    hacer los siguientes algoritmos:

    \begin{algorithm}
        \begin{algorithmic}
            \STATE $\pi_{finally}(x, G(D))$
            \STATE Donde $x$ es una cadena en $L(TM)$
            \STATE $x \leftarrow (x_{1}\mathellipsis x_{n})$
            \FORALL {$r \leftarrow 1$ \textbf{to} $m$}
               \STATE $\pi(x, G(r))$
               \STATE \textbf{broadcast($v_{0},G(r), M$)}
               \STATE Donde $G(r)$ es una grafica de diametro $r$
            \ENDFOR
        \end{algorithmic}
        \caption{$\pi(x,G)$\label{lss}}
    \end{algorithm}
\end{document}